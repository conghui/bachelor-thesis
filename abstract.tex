\begin{abstract}

  The goal of exploration seismology is to find oil and gas reservoirs by
  seismically imaging the earth’s reflectivity distribution The
  conventional one-way wave equation pre-stack depth migration algorithm is
  relatively fast but not able to identify the potential existence of
  hydrocarbon in complex geology. The Reverse time pre-stack depth
  migration (RTM) , which use the two way acoustic wave equation, offers
  insights into complex geology that were previously impossible to
  interpret or understanding using seismic data.  However it is the most
  computationally-demanding algorithm in the oil and gas exploration which
  was considered impractical for production 3D depth imaging project using
  conventional CPUs. Thus hardware accelerator like GPUs and Field
  Programmable Gate Arrays (FPGAs) emerged and have been used as an
  alternative for the conventional computing architectures (CPUs) in
  science computing application and have shown considerable speed-ups.
  Compared with CPUs, FPGA has a better performance but low power
  consumption where logics and algorithms are coded directly into hardware,
  however, only a few applications are implemented on FPGA because the
  steep learning curve for the developers. This thesis, this work, presents
  a solution that takes advantages of FPGA's flexibility to explore
  efficiently data reuse for the 3D Reverse Time Migration algorithm, the
  seismic modeling. Compared with the CPU implementation using two
  quad-core Intel i7 CPUs, the FPGA-based solution without any
  optimization achieves 6x speed-ups on one Xilinx V6-SXT475 FPGA.

  \vspace{4mm}

  \noindent{\emph{Keywords:} \textbf{Reverse Time Migration (RTM); Field
  Programmable Gate Array (FPGA); Stencil; Performance}}
\end{abstract}
