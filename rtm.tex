\section{RTM Algorithm and Analysis}

As it is mentioned in the previous section that the Reverse Time Migration
algorithm can gain a big improvement but it is computationally demanding.
In this section, we have a look at the detail of the algorithm and present
a computation complexity analyze of it.

The main steps of the algorithm is explained as below\cite{rtm}.

\begin{enumerate}
\item the forward extrapolation of a modelled source waved for each shot
  location through a gridded velocity model is performed. And the wave
  field at each time step is saved for later application of the ``imaging
  condition''.
\item the receiver wave field for each shot, which is recorded in the
  field, is backward propagated in time through the same velocity model.
\item at each time step, the corresponding source and receiver wave fields
  are correlated by applying the imaging condition. Thus, the final
  wave field in the source propagating scheme is correlated with the
  initial wave field in the receiver propagation scheme, and so on backward
  through the receiver propagation.
\item the result are summed to form a partial image volume for each shot.
\item the image volumes for consecutive gathers are spatially summed to
  produce the final pre-stack depth image.
\end{enumerate}

So, now you may know why RTM is so computation expensive and has several
decades to be commercially implemented.

\subsection{Mathematical Derivation of RTM}

If we assume that the wave that the wave injects is acoustic wave (sound
wave), then we can use the acoustic wave equation as the first step of the
derivation.

\begin{equation}
  \frac{1}{c^2}\cdot\frac{\partial ^2u}{\partial t^2}=\bigtriangledown ^2u +s
\end{equation}

