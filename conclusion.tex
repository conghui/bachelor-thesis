\section{Conclusion And Outlook} % (fold)
\label{sec:Conclusion}

This thesis presents Field Programmable Gate Array (FPGA) designing with
the help of Maxeler OS and Reverse Time Migration (RTM) algorithm, which is
the most computationally demanding algorithm in oil and gas exploration
operations. In addition, the thesis presents FPGA-based solution for the
RTM algorithm.

By utilizing the reconfigurability of the FPGA device, my RTM design
manages to speed up the calculation in term of the hardware calculation
instead of software calculation. When comparing to Intel(R) Core(TM) i7
CPU, the naive implementation of FPGA achieves 6X speed up without any
optimization.

I think FPGA is becoming efficient High Performance Computing solution,
which is an alternative to conventional CPU and GPU, because it has
abundant computation and storage resources on the board. What more,
paralleling the FPGA kernel by utilizing more FPGA resource is a good way to
optimize the FPGA kernel, which will result in better performance.
Meanwhile, users can provide other boost to the performance as they can
perform customization on various aspect of the design. These unique
properties makes FPGAs a preferable choice for many
computationally-intensive application.

The current solution based on FPGA for the RTM algorithm is of no
optimization, and treat the wave field of boundary as zeros. More work can
be done on the optimization on FPGA, which will increase the usage of it
and  increase its performance. In addition, some tools can be used to draw
the image from the RTM output, which gives us a more intuitive insight of the
structure in the interior of the earth.
% section Conclusion (end)
